\documentclass{report}

\begin{document}

 
\thispagestyle{empty}

\def\lskip{\vspace{0.5cm}}

\begin{center}
ÉCOLE CENTRALE DE NANTES
\end{center}

\vspace{2cm}

% CORO-IMARO students
\begin{center} \large\sc MASTER CORO-IMARO\\ \normalsize{``CONTROL and ROBOTICS''} 
\end{center}

\begin{center}
	2019 / 2020\\
	\lskip
	Master Thesis Report
	\lskip
	
	Presented by \lskip 
	
	MA JIE \lskip
	
	January 2019 \lskip\lskip
	
	{\Large \textbf{One strategy for implementing EM planner with limited computing capability}}
	\vfill
Jury \lskip
		
	\end{center}
	


\begin{tabular}{p{3cm}p{7cm}p{5cm} }
 % President: & Name & Position (Institution) \\ & & \\     % for final defense only (not bibliography)
 Evaluators: & Ina Taralova & Maitre de conferences(ECN) \\
	      & Olivier Kermorgant & Professor(LS2N,ECN) \\ 
  Supervisor(s):  & BAI YU & Engineer (Mei Tuan,china) \\		  
\end{tabular}

\lskip

%\begin{flushleft}
%\end{flushleft}
%ctrl + t  or + u
%\newpage
%\addtocounter{page}{-1}
%\pagestyle{fancy}
\newpage
\addtocounter{page}{-1}
\pagestyle{fancy}
  
 
  \section*{Abstract}
% Do not forget to check each reference while importing in your Bibtex file.
% Especially, IEEExplore export may lead to ill-formatted conference name like \emph{Robotics and Automation, 
% IEEE International Conference on}.
\par An Internet company - Baidu is working on the autonomous driving, they have introduced a local path planning algorithm - expectation maximum planner(EM planner), this is a sampling-based algorithm. We hope to apply this algorithm to logistics robots to achieve the purpose of automatic delivery.\\
\indent
Compared with logistics robots, autonomous vehicles have higher requirements for various algorithms. Autonomous vehicles need to face more complex road environments, and they must be able to respond quickly when driving at high speeds.
We can apply the algorithm to the logistics robots, but considering the cost of logistics robots, their motherboards have lower computing capability, if we use the exhaustive search approach (brute-forced search), the EM planner can always make an ideal choice based on the cost function. However, due to the limited computing capability of the motherboard, we need to adopt appropriate strategies to make the right behavior quickly and well.\\
\indent
The objective of this bibliography report is to present the state-of-the-art work on EM planner, with focus on details of algorithm implementation.


 \newpage
 
 \section*{Acknowledgements}
 First of al, I would like to thank Prof. Olivier Kermorgant agreed to my request to return to China for internship. \\
 \indent
 I am thankful to the Meituan can give me a great opportunity to work on a topic of my interest.
 
 \newpage
 
 \section*{Notations}
 
 \newpage
 
  \section*{Abbreviations}
 
 \newpage
 
 \listoffigures
 
\listoftables
 
 \tableofcontents
 
 
 \chapter*{Introduction}
 \addcontentsline{toc}{chapter}{Introduction}	 % non-numbered chapters do not appear in table of contents by default
 
 
 \chapter{State of the art}
 
 \section{First topic}
 
 \section{Second topic}
 
 \chapter{Actual work}
  
 
 When dealing with rectangled triangles (see Figure \ref{triangle}) I sometimes used this theorem from \cite{pythm001}:
 \begin{equation}\label{theo}
  a^2 + b^2 = c^2
 \end{equation}The demonstration is in Appendix \ref{sec:prooftheorem}.
 
 
 \chapter{Experiments}
 
 When trying to draw a rectangled triangle, my program comes up with Figure \ref{triangle2} that is neither rectangled nor a triangle.
 
Unless there is a bug in my program, which is unlikely, this research indicates that the whole theory on triangles having 3 sides has been wrong for years, maybe decades.
 
 
 \chapter*{Conclusion}
 \addcontentsline{toc}{chapter}{Conclusion}
 
 % switch to A-B-C chaptering
 \appendix	
 
 \chapter{Proof of theorem \ref{theo}}
 \label{sec:prooftheorem}
 
 
 
 \addcontentsline{toc}{chapter}{Bibliography}

 \bibliographystyle{IEEEtran}
 
 \bibliography{../biblio}
 
 
 
 
\end{document}
